\documentclass[a4paper]{article}

\usepackage[T1]{fontenc}
\usepackage[utf8]{inputenc}
\usepackage[french]{babel}
\usepackage{mathpazo}
\usepackage[scaled]{helvet}
\usepackage{courier}
\usepackage[sf,bf]{titlesec}
\usepackage[margin=0.75in]{geometry}
\usepackage{tabularx}

\makeatletter
\newenvironment{expl}{%
  \begin{list}{}{%
    \small\itshape%
    \topsep\z@%
    \listparindent0pt%\parindent%
    \parsep0.75\baselineskip%\parskip%
    \setlength{\leftmargin}{20mm}%
    \setlength{\rightmargin}{20mm}%
  }
    \item[]}%
    {\end{list}}
\makeatother

% Indiquez votre prénom (en minuscules) et votre nom (en majuscules)
\title{Rapport de TP \\ Architecture des systèmes d'exploitation}
\author{\underline{Paul} \underline{MEYER}}
\date{Novembre 2021}

\begin{document}

  \maketitle

% Mettez une table des matières si votre rapport contient plus
% de 3 pages ou si vous ne suivez pas le plan suggéré :
%\tableofcontents

% Dans votre rapport final, supprimez toutes les explications
% (c'est-à-dire tous les environnements \begin{expl} ... \end{expl}).

  \section{Introduction}

  \begin{expl}
    L'objectif de ce TP à rendre est d'implémenter un vaccinodrome qui prend en charge des patients, venant se faire
    vacciner par des médecins. Ces médecins vaccinent les patients dans des box. Un patient attendra à l'extérieur jusqu'à
    ce qu'il y ait une place dans la salle d'attente. Dans celle-ci il attendra sur un siège numéroté, qu'un médecin soit disponible. Lorsqu'un médecin sera disponible, il entrera dans le box pour se faire vacciner, puis partira. Les médecins attendrons jusqu'à la fermeture du vaccinodrome pour partir.

    Dans cette implémentation, il a été choisi de laisser le patient choisir un box disponible.
    Lorsqu'un médecin est disponible, le patient en sera informé et cherchera le box disponible.
    Il monopolisera le box pour toute la durée de la vaccination. Une fois le box monopolisé, le patient notifiera le médecin qu'il veut être vacciné, puis attendra que le médecin le laisse partir. En partant le box sera libéré et un nouvelle notification de disponibilité du médecin sera envoyée aux patients.
  \end{expl}

  \section{Structure de données}

  \subsection{Structures de données partagées}\label{sec-shm}

  \begin{expl}
    J'ai utilisé un total de 3 structures, toutes font partie de la mémoire partagée.

    La structure principale s'appelle vaccinodrome\_t et contient toutes les données utiles,
    sémaphores, variables globales.

    \begin{tabularx}{\linewidth}{|l|l|l|X|}
      \hline
      % \multicolumn{4}{|l|}{Distributeur de gel
      %   (\texttt{struct distr\_gelha})}
      % \\ \hline
      Champ & Type & Init & Description \\ \hline%
      nbSieges & int & -- & Nombre de sièges dans le vaccinodrome. \\ \hline%
      nbMedecins & int & -- & Nombre de médecins dans le vaccinodrome. \\ \hline%
      temps & int & -- & Temps pris par une vaccination en ms. \\ \hline%

      waitingRoom & asem\_t & nbSieges & Sémaphore qui gère l'accès a la salle d'attente. \\ \hline%
      medecinDisponibles & asem\_t & 0 & Sémaphore qui notifie lorsqu'un médecin est disponible. \\ \hline%
      fermer & asem\_t & 0 & Sémaphore qui attends la fermeture de tous les médecins. \\ \hline%

      asemMutex & asem\_t & 1 & Sémaphore binaire qui sert de "mutex" lors d'un accès a la mémoire partagée box\_t. \\ \hline%
      siegeMutex & asem\_t & 1 & Sémaphore binaire qui sert de "mutex" lors d'un accès a la mémoire partagée siege\_t. \\ \hline%
      waitingMutex & asem\_t & 1 & Sémaphore binaire qui sert de "mutex" lors d'un accès a la variable currPatientWaiting. \\ \hline%

      currMedecins & int & 0 & Nombre médecins actuellement dans le vaccinodrome. \\ \hline%
      currPatientWaiting & int & 0 & Nombre de patients attendant à l'extérieur du vaccinodrome. \\ \hline%
      statut & int & 0 & Statut de fermeture/ouverture du vaccinodrome (0 ouvert, 1 fermé) \\ \hline%
      medecins & box\_t[] & -- & Flexible array contenant les boxes de médecins \\ \hline%

    \end{tabularx}

    \newpage

    La seconde structure s'appelle box\_t, et contient toutes les informations d'un box.

    \begin{tabularx}{\linewidth}{|l|l|l|X|}
      \hline
      % \multicolumn{4}{|l|}{Distributeur de gel
      %   (\texttt{struct distr\_gelha})}
      % \\ \hline
      Champ & Type & Init & Description \\ \hline%
      demandeVaccin & asem\_t & 0 & Sémaphore qui attends qu'un patient demande un vaccin au médecin. \\ \hline%
      termineVaccin & asem\_t & 0 & Sémaphore notifiant le patient lorsqu'il a été vacciné. \\ \hline%
      status & int & 0 & Statut du box (0 libre, 1 occupé, 2 fermé) \\ \hline%
      medecin & int & -- & Nombre de médecins dans le vaccinodrome. \\ \hline%
      patient & char[] & -- & Nom du patient pris en charge \\ \hline%

    \end{tabularx}

    La troisième structure s'appelle siege\_t, et contient toutes les informations d'un siège.

    \begin{tabularx}{\linewidth}{|l|l|l|X|}
      \hline
      % \multicolumn{4}{|l|}{Distributeur de gel
      %   (\texttt{struct distr\_gelha})}
      % \\ \hline
      Champ & Type & Init & Description \\ \hline%
      siege & int & -- & Id (numéro) du siège \\ \hline%
      statut & int & 0 & Statut du box (0 libre, 1 occupé) \\ \hline%

    \end{tabularx}

  \end{expl}

  \subsection{Structures de données non partagées}

  \begin{expl}
    Aucune donnée utile n'est pas partagée.
  \end{expl}

  \section{Synchronisations}

  \begin{expl}
    Chacune des sous-sections suivantes doit décrire un aspect de la
    synchronisation. Elles ont toutes la même organisation, et doivent
    contenir deux ou trois parties clairement délimitées.

    \begin{enumerate}

      \item La première partie décrit tous les objets impliqués dans la
      synchronisation de l'accueil des patients dans le vaccinodrome. Il
      s'agit des données «~normales~» (par exemple un entier) et des
      sémaphores. Pour chacun de ces éléments, précisez si il est
      rattaché au vaccinodrome globalement ou à un patient, ou encore à
      un médecin. Par exemple, si le vaccinodrome était muni d'un
      distributeur de gel hydroalcoolique grâce auxquels les patients
      peuvent se nettoyer les mains, il faudrait mentionner~:

      \begin{tabularx}{\linewidth}{|l|l|>{\strut}X|}
        \hline%
        sm\_dg & asem\_t & le sémaphore représentant le nombre de
        distributeurs \underline{du vaccinodrome} (voir
        section~\ref{sec-shm}) \\ \hline%
        monid & int & le nom %
        \underline{\raisebox{0pt}[\height][0pt]{du patient}} \\ \hline%
      \end{tabularx}

      \item La seconde partie doit contenir un pseudo-code de ce
      qu'exécute \emph{chaque} acteur impliqué (le patient, le médecin,
      autres). Seuls les aspects importants doivent être présentés~:
      opérations sur les sémaphores, affectation de valeurs aux
      variables partagées... Le reste (par exemple les messages liés à
      \texttt{DEBUG\_OUTPUT}) ne doivent pas apparaître ici. Dans
      l'exemple des distributeurs de gel, le pseudo-code serait~:

      \begin{verbatim}
// Ce code est exécuté par le patient
P (sm_dg)
se_laver_les_mains (monid)
V (sm_dg)
      \end{verbatim}

      \item La troisième partie contient toutes les remarques que vous
      jugez pertinentes. Vous pouvez par exemple y préciser les
      conditions de concurrence, ou les éventuelles limitations ou
      propriétés remarquables de votre solution.

      Dans l'exemple des distributeurs de gel, vous pourriez préciser
      que vous avez pensé à implémenter à la place une attente limitée
      dans le temps pour les patients qui portaient des gants en entrant
      (en expliquant comment vous auriez fait), mais que cela aurait
      compliqué le code pour un intérêt limité.
    \end{enumerate}

    Vous pouvez ajouter des sous-sections si vous avez d'autres
    synchronisations importantes, ou fragmenter celles qui vous sont
    proposées ci-dessous.
  \end{expl}


  \subsection{Arrivée d'un patient}\label{arrivee-patient}

  \begin{expl}
    Expliquez dans cette sous-section, conformément aux indications
    ci-dessus, tout ce qui se passe lorsqu'un patient arrive,
    \emph{avant} qu'il interagisse avec un médecin (si le vaccinodrome
    est ouvert) ou reparte. Mentionnez toutes les informations
    partagées qui sont utilisées et toutes les synchronisations
    nécessaires.
  \end{expl}

  \subsection{Arrivée d'un médecin}

  \begin{expl}
    Expliquez dans cette sous-section tout ce qui se passe lorsqu'un
    médecin arrive, \emph{avant} qu'il interagisse avec un patient.
  \end{expl}

  \subsection{Interactions entre patients et médecins}

  \begin{expl}
    Expliquez ici comment patients et médecins sont «~mis en contact~».
    Expliquez la stratégie que vous avez choisie, c'est-à-dire comment
    1) soit un patient trouve un médecin dès qu'un médecin est
    disponible, 2) soit un médecin trouve un patient dans la salle
    d'attente. Expliquez également ce qui se passe entre le moment où un
    médecin et un patient sont mis en contact, et le moment où le
    patient vacciné peut quitter le vaccinodrome et le médecin s'occuper
    d'un (éventuel) autre patient.
  \end{expl}

  \subsection{Fermeture du vaccinodrome}

  \begin{expl}
    Expliquez les détails de ce que fait le programme \texttt{fermer}
    (et seulement ce programme), y compris lorsqu'il reste des patients
    et/ou des médecins.
  \end{expl}

  \subsection{Patients après fermeture}

  \begin{expl}
    Expliquez ce qui se passe pour un patient assis dans la salle
    d'attente après la fermeture. (Notez que le cas du patient qui
    arrive \emph{après} la fermeture doit avoir été décrit dans la
    section~\ref{arrivee-patient}.)
  \end{expl}

  \subsection{Médecins après fermeture}

  \begin{expl}
    Expliquez ce qui se passe pour un médecin après la fermeture. (Le
    cas du médecin qui arrive \emph{après} la fermeture doit avoir été
    décrit dans la section~\ref{arrivee-patient}.)
  \end{expl}

  \section{Remarques sur l'implémentation}

  \begin{expl}
    Placez ici toute remarque relative à vos choix d'implémentation (si
    vous en avez). Par exemple~: si vous avez placé des temporisations
    particulières pour vérifier que certains cas de concurrence étaient
    gérés correctement, expliquez cela dans cette section. Toute autre
    remarque qui aide à comprendre votre code est à placer ici.
  \end{expl}

  \section{Conclusion}

  \begin{expl}
    Tirez les conclusions de votre projet~: limites de votre
    implémentation, difficultés particulières, subtilités dont vous êtes
    fiers, etc.
  \end{expl}

\end{document}
